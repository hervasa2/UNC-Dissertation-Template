\chapter{Introduction}

Neutrinoless double-beta decay (\novbb{}) is a possible, extremely rare decay which, if observed, would prove that the neutrino is its own antiparticle. To detect such a rare decay, \novbb{} experiments are built deep underground, shielded from cosmic radiation, with ultra-radiopure materials which have undergone extensive assays. An exciting candidate isotope to search for \novbb{} is \geEn{}. Germanium detector technology has been developed for decades, finding applications in radiometric assays and nuclear non-proliferation.  A trans-Atlantic union of the germanium-based experiments, {\MJDEMit} and GERDA, gave birth to the LEGEND collaboration. The Large Enriched Experiment for Neutrinoless Double-beta Decay is pursuing a tonne-scale \geEn{} experiment, with \novbb{} discovery potential at a half-life approaching, or at, $10^{28}$ years~\cite{LEGEND2021}.

Together with a team at the Max Planck Institute for Physics in Munich, a novel Compton scanner was commissioned~\cite{compton_scanner}. The principle of operation of a Compton scanner is the same as of a conventional camera. By capturing light -- or in this case gamma-rays -- scattering off a subject, an image is reconstructed. The position sensing capabilities of the scanner allows for the bulk of large volume-detectors to be characterized. A 2\,kg Inverted Coaxial Point-Contact (ICPC) detector -- the principal detector technology used in LEGEND -- was deployed in the Compton scanner in summer 2021. This thesis captures the main results of this measurement campaign.

A theoretical foundation of gamma-ray interactions, germanium detectors, and \novbb{} is provided in Chapters~\ref{chap:gammas}--\ref{chap:theory}. The latter focuses on the experimental techniques used to search for \novbb{} and leads into an overview of the LEGEND experimental program in Chapter~\ref{chap:legend}. The working principle of the Compton scanner is expanded on in Chapter~\ref{chap:scanner}, where an extensive description of the instrument is provided. The instrumentation had to be slightly modified to successfully integrate the ICPC in the system, which was deployed in its vendor cryostat. 

Multiple radioactive sources were used in the measurement campaign. The state-of-the-art analysis techniques developed by {\MJMit} and GERDA, were applied to ICPC data through a new \julia{}-based software package tailor-made for this work. Chapter~\ref{chap:param} shows the results from this standard set of analysis steps, including energy estimation, charge trapping correction, and multi-site event classification. Additionally, to manage the high rates of the Compton scanner source, a new pileup discriminator was developed. 

The operating principle of the Compton scanner is applied to data in Chapter~\ref{chap:pipeline}, where the reader is taken through the analysis steps required to perform position reconstruction and build a pulse-shape library. The capabilities of the Compton scanner are further capitalized on to search for regions of high charge trapping and investigate the topology of the depletion surface. These results are shown in Chapters~\ref{chap:trapping}--\ref{chap:bubble}. Depletion surface evolution is used to construct an impurity model in the following chapter, where results from simulation are compared to data. The thesis concludes with the pulse shape library, which draws inputs from all the analysis and simulation steps that precede it. 