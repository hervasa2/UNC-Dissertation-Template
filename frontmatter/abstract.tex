%The word �Abstract� should be centered 2? below the top of the page. 
%Skip one line, then center your name followed by the title of the 
%thesis/dissertation. Use as many lines as necessary. Centered below the 
%title include the phrase, in parentheses, �(Under the direction of  
%_________)� and include the name(s) of the dissertation advisor(s).
%Skip one line and begin the content of the abstract. It should be 
%double-spaced and conform to margin guidelines. An abstract should not 
%exceed 150 words for a thesis and 350 words for a dissertation. The 
%latter is a requirement of both the Graduate School and UMI's 
%Dissertation Abstracts International.
%Because your dissertation abstract will be published, please prepare and 
%proofread it carefully. Print all symbols and foreign words clearly and 
%accurately to avoid errors or delays. Make sure that the title given at 
%the top of the abstract has the same wording as the title shown on your 
%title page. Avoid mathematical formulas, diagrams, and other 
%illustrative materials, and only offer the briefest possible description 
%of your thesis/dissertation and a concise summary of its conclusions. Do 
%not include lengthy explanations and opinions.
%The abstract should bear the lower case Roman number ii (if you did not 
%include a copyright page) or iii (if you include a copyright page).

\begin{center}
\vspace*{52pt}
{\normalfont\textbf{ABSTRACT}}
\vspace{11pt}

\begin{singlespace}
DAVID HERVAS AGUILAR: Characterizing Bulk Signals in an Inverted Coaxial Point-Contact Detector to Inform Rare-Event Searches \\
(Under the direction of John F. Wilkerson)
\end{singlespace}
\end{center}

A novel HPGe Compton scanner was constructed at the Max Planck Institute for Physics in Munich. In this apparatus, highly penetrating gamma-rays deposit energy in the Ge lattice, inducing a signal coincident with that of a position-and-energy-sensitive camera. Position reconstruction in large-volume Ge detectors with a millimeter level resolution and reasonable detector scanning times was demonstrated. The scanning apparatus was employed to characterize the bulk of a 2\,kg Inverted Coaxial Point-Contact detector. It was hypothesized that in such large volume detectors, deep hole trapping in the bulk could lead to the severe energy degradation of signals. At 95 K, no evidence of significant deep hole trapping was found. This is of importance for the next generation of Ge-based neutrinoless double-beta decay experiments, where at the tonne-scale even small underperforming volumes could populate the signal window with energy degraded signals. Using the Compton scanner, the first experimental images of the depletion surface of a large-volume, non-segmented HPGe detector were produced. It was shown that a modified impurity model of the detector, determined using the evolution of the depletion surface at different biases, outperforms the conventional model of the impurity profile. Pulse-shape simulation is heavily reliant on the impurity model. Thus, a precise understanding of impurities is critical for the development of background rejection techniques used in Ge-based rare-event searches.

\clearpage
