\chapter{Applications to LEGEND and Conclusions} 

Neutrinoless double-beta decay poses an exciting way of probing the absolute neutrino mass and the Majorana nature of the neutrino.  Regardless of the mechanism involved in its production, the observation of \novbb{} will imply new physics, exhibiting lepton number violation, and providing insight into the matter-antimatter asymmetry in the universe. A multi-isotope tonne-scale experimental program for \novbb{} has been proposed. The next generation of experiments will be able to cover the entire inverted ordering region, providing limits 2 -- 3 orders higher than the current limits if no discovery is made. \geEn{} detector technology is extremely well suited for this challenge. The US and Italy based experiments, the {\MJDEMit} and GERDA respectively, exploited this technology to search for \novbb{} in \geEn{}-enriched detectors. GERDA has set the strongest limit on the half-life of this decay, above $1.8 \times 10^{26}$ years, and both experiments have the lowest backgrounds of any experiment in the field. The combination of these complimentary technologies has already shown promise in the first phase of the next generation experiment in \geEn{}, LEGEND. The LEGEND collaboration used existing infrastructure to expedite physics results, and the data taking phase of LEGEND-200 is underway. 

With the unprecedented predicted background rate of LEGEND-1000 and 10\,tonne\,yr of exposure, only four signal counts would constitute a 3$\sigma$ discovery of \novbb{}. An understanding of pulse formation is critical to reach this background goal. As Inverted Coaxial Point-Contact detector technology continues to mature, precise modeling of the impurity profile throughout the bulk of new detectors will inform background discrimination techniques. To characterize the bulk of such detectors a 2\,kg ICPC was integrated with the Compton scanner at the Max Planck Institute for Physics. It had been previously demonstrated that the apparatus can achieve millimeter level position reconstruction resolution. The novel application of a position-and energy sensitive pixelated camera drastically reduced scanning times compared to traditional slit collimation techniques. This capability allowed for the bulk of the ICPC to be scanned at multiple crystal axes and biases in a matter of weeks once successfully integrated. 

The integration itself faced many unexpected challenges, including noise-bursts cased by the simultaneous operation of the camera and detector. These issues were resolved through upgrades to the Compton Scanning system itself. The optical upgrade -- which used optocouplers in the synchronization lines of the camera -- proved instrumental to the creation of a pulse shape library of the ICPC, where noise reduction played a critical roll in data quality. Future detector deployments in the scanner will also greatly benefit from the optical synchronization technique deployed. By exploiting the timing properties of the system, a measurement of $t_0$ bias for pulses in the top of the detector arm was made. Such measurements could benefit from faster optocoupler response, which would provide a more precise synchronization of the camera and detector data streams. To process the data, new a new software package, incorporating the state-of-the-art analysis techniques developed by {\MJMit} and GERDA, was built. This package is already informing the \julia{}-based LEGEND software stack.

Compared to Compton scans of smaller detectors, the validated reconstructed events of the ICPC contained a much greater contamination of multi-site events. Although an $AvsE$ cut successfully removed such events from the validated population, additional studies are required to understand why such events were validated in the first place. Such studies have the potential of improving the position reconstruction algorithm itself and would be crucial to the study of detectors as large as those envisioned for LEGEND-1000. 

It was hypothesized that in such large volume detectors, deep hole trapping in the bulk could lead to the severe energy degradation of signals. It was demonstrated that at 95\,K, and operating in a vacuum cryostat, the energy recorded by the ICPC falls within the confines of the established energy resolution models for gammas impinging throughout the bulk. Therefore, no evidence of significant deep hole trapping was found. However, further studies are required to determine that this holds at lower temperatures where charge trapping effects are greater. In the envisioned tonne-scale \geEn{} experiment, detectors with small underperforming volumes could populate the \novbb{} signal window with energy degraded signals. 

LEGEND is currently pushing the limits of ICPC technology to new frontiers. To ensure detector performance and inform simulations, the techniques described in this work could form part of a suite of detector characterization procedures. It was demonstrated that an impurity model informed by depletion surface evolution, produced a better agreement between simulated and measured pulse-shape in comparison with the conventional model based on vendor and depletion information alone. To achieve this, the first experimental images of the depletion surface of a large-volume, non-segmented germanium detector were produced. Using the new impurity model, it was found that simulated pulses were approximately 10\% faster than those measured along the $\left<1\,0\,0\right>$-axis of the detector. This finding -- along with a similar Compton scanner result from a segmented p-type detector -- hints that the hole drift velocities determined by Bruyneel \textit{et al.} could be 10\% too fast. Nevertheless, uncertainties in the impurity profile, temperature dependence and electronics response model are too large to make a decisive claim.

Similar approximations of the impurity profile of ICPCs can be attained by comparing measured and simulated CV curves. Although such measurements can be made in a fraction of the time as depletion surface imaging, they require extensive instrumentation of the detector, rendering the technique impractical for broad use in LEGEND detector characterization. On the other hand, similar to the ICPC studied, LEGEND detectors can be deployed in the Compton scanner within the vendor cryostat. In this scenario, and with the current instrumentation, the depletion surface at a given bias can be imaged in approximately three days. Future upgrades to the Compton scanner, in particular incorporating additional camera modules, can cut down this time significantly. In this manner, the bulk of LEGEND detectors can be characterized -- providing data for impurity model fits -- in practical timeframes. 

The observation of \novbb{} would give insight to many of the fundamental questions in physics, including lepton number conservation and the nature of the neutrino itself. This is truly transformational science and its ripple effects would span disciplines. The extremely long half-lives involved in \novbb{} present a tremendous experimental challenge; however, the time is ripe for tonne-scale experiments, such as LEGEND, to tackle this immense undertaking.